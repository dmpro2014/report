\begin{table}
%\centering
\begin{tabular}{ | L{2cm} | L{3cm} | L{3cm} | L{3cm} | L{3cm} |}
\hline
\multicolumn{5}{|c|}{Solder and test plan} \\
\hline 
Step & Components & Description & Expected Results & Actual Result \\\hline
5V power & USB connector & Measure voltage on the header & Measure 5 volts on the incoming power line. & \\
\hline
 & & & &\\
3.3V power & 3.3V header, 3.3V LDO, voltage regulator capacitors, power indicator LED and ARM programming header  & Solder the component and measure the voltage on the voltage-pin and observe the  LED & Observe the power indicator LED light up and measure 3.3V on the programming header. & TBA \\
\hline
 & & & &\\
1.2V & 1.2V LDO, power indicator LED, capacitor, resistor. & Solder the components and measure the BGA solder pads for 1.2V and observe the LED & 1.2V on BGA solder pads, power indicator LED lights up. &\\
\hline
 & & & &\\
BGA & EFM32GG990, LX45 & Place the components on a new PCB and bake them in the oven & Components do not fall off and all pins are connected. &\\
 \hline
 & & & &\\
Connecting FPGA and MCU to power & Power supply, LDO voltage regulator, capacitor & Reconnect the electrical circuit on the new board & The FPGA and MCU get the needed power. &\\
\hline
& & & &\\

Critical components & Oscillator, high and low frequency crystal & Solder the major components and connected header. Then use a logic analyzer to check the outputs.  & Consistent pulses from the clocks and oscillator. & \\

\hline
& & & &\\
The rest & The remaining components & Solder the components  & A working PCB. &\\
\hline


\end{tabular}
\caption{\label{tab:widgets}Solder plan.}
\end{table}

