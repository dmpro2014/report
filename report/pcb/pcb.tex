\chapter{PCB}
\label{sec:pcb}

This chapter will describe the PCB layout, the main components and the design philosophy that went into solving the system requirements.
%\section{Layout Overview}

\section{Design process}
Split project schematics into several connected docs for the sake of modularity. Found some application notes and some hardware considerations. Found some components and footprints that would fit...ish with these docs. Then put out the components on the board digitally and routed the wires.


\section{Design for Redundancy}

A lot of effort has gone into making sure that if something fails, there is a backup plan.
This means that most of the wires are connected to jumpers, which can be rerouted manually.
Because of this, the board is not optimized for size, but was rather made to optimize for
highest possible chance of success.

In addition, small graphic that goes 
\begin{verbatim}
-------------------
| Backup solution |
|                 |
|  -------------  |
| | Core design | |
| |             | |
| |             | |
| --------------- |
|                 |
-------------------
 \end{verbatim}

\section{Components}

\subsection{Microcontroller / System Control Unit}
The EFM32 Giant Gecko 32-bit Microcontroller from Energy Micro was chosen as the controller for this project.
A microcontroller from Energy Micro was required for the task and this particular controller is
very energy efficient, which is a plus.
In addition to this, there were a lot of development boards available,
plus over half of the group had experience with this controller from the subject
TDT4258 Energy Efficient Computer Systems.

The EFM32GG990F512-BGA112 was chosen as a powerful enough version of the microcontroller.
Not only did it have enough GPIO pins for our needs, but it could....something %TODO: skriv mer om dette

\section{Input}
Was solved through usb for data communication between microcontroller and host PC, plus a serial(RS-232) backup scheme.

\section{Output}
Main scheme was HDMI connector. Backup: HDMI headers with gutted cable connected, then $\mu$-VGA module conected to the FPGA and finally $\mu$-VGA module connected to the microcontroller.

\section{Power}
usb connection with successive headers that allow for using an external power supply in case power fails.

\section{Bus}
Standard EBI connection between FPGA and MCU, but with headers in between such that external connection can be done in case of failure on one side of the connection, as well as an easy way to check transmitted signals during debugging.

\section{Memory}
Same as with bus. Connected to FPGA via headers.
\section{Clocks}
FPGA oscillator along with header on which an external oscillator resource can be connected by way of backup.
The microcontroller clocks on the other hand have a backup solution in that the microcontroller has internal RC-oscilators for use in case the crystals malfunction.

\section{Footprints}
Predominantly SMDs(surface mounted device) large enough for humans to solder(with a couple of exceptions)
...Uncertain if we neeed a 'Footprint section'

% where should this be placed?
\documentclass[../main/report.tex]{subfiles}
\begin{document}

\subsection{HDMI}

From a demo makers perspective, a GPU without a video output is commonly known as a space heater.
Demolicious uses HDMI for its video output, making it easy to connect to any recent video display.

HDMI is a streaming protocol; the receiver reads data from the cable at a fixed rate.
In Demolicious, the GPU has priority access to the memory.
This means that the video unit may not have access to the framebuffer (which lies in memory) when it's time to send a pixel.
To alleviate this issue, as much of the framebuffer as possible is prefetched into a buffer whenever memory is idle.
Should the buffer underflow, sending of late pixels must be abandoned.
Otherwise, pixels will not be synchronized with the position they should appear at on the screen.

Control signals assert where in the data stream a new frame of video starts and ends.
These allow the receiver to determine the resolution and refresh rate of the video.

The lowest resolution supported by HDMI is 640x480.
As this is larger than our framebuffers (64 x 64 pixels), a letterbox is added around the picture.
For debugging purposes, the letterbox consists of a low-contrast checker pattern.

To actually send the data over HDMI, control signals and pixel data are split into three channels.
They are then encoded using a scheme known as TMDS.
The purpose of TMDS is to minimize the effect of noise over the physical connection.

TMDS uses 10 bits to encode either an 8-bit color value when sending a image, or control values when not.
Demolicious uses a 16-bit word size, so colors are represented with 5 bits for red, 6 for green and 5 for blue.
These are resized to 8-bit values using a scheme that allows for both complete black and white colors.
Each channel is then serialized before being output together with a clock using differential-signaling.

Finally, to avoid a visual artifact known as \emph{screen tearing}, a technique known as \emph{V-sync} with \emph{double buffering} is used.
These techniques ensure that only complete frames of video are output, increasing visual fidelity.

\end{document}

