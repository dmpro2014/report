\chapter{PCB}
\label{sec:pcb}

This chapter will describe the PCB layout, the main components and the design philosophy that went into solving the system requirements.
%\section{Layout Overview}
\section{Physical system structure}

\missingfigure{Physical overview}


\section{Design for Redundancy}

When designing a pcb for a system, there is no easy way to correct mistakes.
A wire can not be rerouted if there is a problem.
Becayse of this, a lot of effort has gone into making backup plans if something fails.
Each part of the pcb needs to be able to work without the rest of the pcb.
This is done by having jumpers on each section, which can be rerouted manually.
That way each part can be connected to other parts of the board, or to other sources.
Because of this, the board is not optimized for size, but was rather made to optimize for highest possible chance of success.

This small graphics explains the design philosophy of the pcb
\begin{verbatim}
     -----------------
    | Backup solution |
    |                 |
    |  -------------  |
    | | Core design | |
    | |             | |
    | |             | |
    | --------------- |
    |                 |
     -----------------
\end{verbatim}

\section{Main Components}

\subsection{Microcontroller / System Control Unit}
The EFM32 Giant Gecko 32-bit Microcontroller from Energy Micro was chosen as the controller for this project.
A microcontroller from Energy Micro was required for the task and this particular controller is
very energy efficient, which is a plus.
In addition to this, there were a lot of development boards available,
plus over half of the group had experience with this controller from the subject
TDT4258 Energy Efficient Computer Systems.

The EFM32GG990F512-BGA112 was chosen as a powerful enough version of the microcontroller.

\subsection{FPGA}
The XC6SLX45-2CSG324I FPGA of the Spartan-6 family from Xilinx was chosen as the FPGA.
This particular FPGA has been used for different tasks on the university before, and the support systems are therefore available to us.
The version was the one used on the PCB.
A less powerful version of this one was available for testing on development boards in the lab.

\section{Input}

The main source of data communication between microcontroller and a host PC is USB.
This protocol has been used for years on projects like these, with good results. \todo{citation needed}

However, if the usb fails, there are backup plans.
Primarily a serial port (RS-232) has been included to work if the USB should fail.
If this also fails, the wires from the serial port is put on headers, which can be used as GPIO pins.

\section{Output}
The main source of output from the FPGA is an HDMI connector.
This is a novel feature this year, as no previous group has tried to implement it before.

Because of this new challenge, a lot of backup schemes were put in place.
The HDMI connector is put on headers, in case the connector fails.
A separate VGA module is connected to the FPGA, in case the HDMI doesn't work and if this fails, a VGA module is also connected to the microcontroller.

\section{Power}
USB connection with successive headers that allow for using an external power supply in case power fails.

\section{Bus}
Standard EBI connection between FPGA and MCU, but with headers in between such that external connection can be done in case of failure on one side of the connection, as well as an easy way to check transmitted signals during debugging.

\section{Clocks}
FPGA oscillator along with header on which an external oscillator resource can be connected by way of backup.
The microcontroller clocks on the other hand have a backup solution in that the microcontroller has internal RC-oscillators for use in case the crystals malfunction.


\section{Footprints}
Predominantly SMDs(surface mounted device) large enough for humans to solder(with a couple of exceptions)

\documentclass[../main/report.tex]{subfiles}
\begin{document}

\subsection{HDMI}

From a demo makers perspective, a GPU without a video output is commonly known as a space heater.
Demolicious uses HDMI for its video output, making it easy to connect to any recent video display.

HDMI is a streaming protocol; the receiver reads data from the cable at a fixed rate.
In Demolicious, the GPU has priority access to the memory.
This means that the video unit may not have access to the framebuffer (which lies in memory) when it's time to send a pixel.
To alleviate this issue, as much of the framebuffer as possible is prefetched into a buffer whenever memory is idle.
Should the buffer underflow, sending of late pixels must be abandoned.
Otherwise, pixels will not be synchronized with the position they should appear at on the screen.

Control signals assert where in the data stream a new frame of video starts and ends.
These allow the receiver to determine the resolution and refresh rate of the video.

The lowest resolution supported by HDMI is 640x480.
As this is larger than our framebuffers (64 x 64 pixels), a letterbox is added around the picture.
For debugging purposes, the letterbox consists of a low-contrast checker pattern.

To actually send the data over HDMI, control signals and pixel data are split into three channels.
They are then encoded using a scheme known as TMDS.
The purpose of TMDS is to minimize the effect of noise over the physical connection.

TMDS uses 10 bits to encode either an 8-bit color value when sending a image, or control values when not.
Demolicious uses a 16-bit word size, so colors are represented with 5 bits for red, 6 for green and 5 for blue.
These are resized to 8-bit values using a scheme that allows for both complete black and white colors.
Each channel is then serialized before being output together with a clock using differential-signaling.

Finally, to avoid a visual artifact known as \emph{screen tearing}, a technique known as \emph{V-sync} with \emph{double buffering} is used.
These techniques ensure that only complete frames of video are output, increasing visual fidelity.

\end{document}


\todo{...Uncertain ie om at vi endra litt på en del footprints, men pga hvordan fysiske komponenter funker,}

\subsubsection{Prebundled}

\subsubsection{Handmade}
Some footprints we had to make ourselves.
This was done inside Altium.
The specification for the footprints was found in the datasheets of the component in question.

When making the handmade footprints some other thoughts were taken into account.

\begin{itemize}
    \item We need to solder these components
    \item The connection on the components is physical, as long as it leads current, it will work.
    \item Some datasheets didn't match the component exactly, but was for a sister component
\end{itemize}

With this in mind we made footprints which was slightly bigger than it needed to.

\begin{verbatim}
Needed:                 Actual:
--------------          --------------
|            |          |            |
|--|      |--|       |--|--|      |--|--|
|  |      |  |       |  |  |      |  |  |
|..|      |--|       |--|..|      |--|--|
|            |          |            |
--------------          --------------
\end{verbatim}
>>>>>>> Add intro footprint
