\documentclass[../main/report.tex]{subfiles}
\begin{document}

\chapter{Physical Implementation}
\label{sec:pcb}

This chapter will describe the PCB layout, the main components and the design philosophy that went into solving the system requirements.
%\section{Layout Overview}
\documentclass[../main/report.tex]{subfiles}
\begin{document}

\section{Physical system structure}

\missingfigure{Physical overview}

\end{document}

\section{Design for Redundancy}

When designing a system the PCB is one of the harder things to debug.
If a wire inside the PCB is wrong, there is not munch that can be done, except buying a new PCB.
This becomes apparant when seeing most projects only have a working PCB on the 3rd of maybe even the 4th try.
Because of this, a strong philosophy has been used in the design for the PCB, as shown in figure \ref{fig:pcb_philosophy}.

To make sure the propability that a pcb will have a working design will be as high as possible, all aspects of the PCB has one or more backup plans.
All important wires and unused pins have been put onto headers, which can be rerouted manually.
That way each part can be connected to other parts of the board, or to other sources.
Because of this, the board is not optimized for size, but was rather made to optimize for highest possible chance of success.

Another part of this design is that each vital part of the PCB follows a modular design.
Each isolated part will work on its own, with the help of other components, if the rest of the board fails.
A PCB with a broken MCU, but with a functioning FPGA can be connected together and work as one system none the wiser.

\section{Main Components}

% Physical Overview
\documentclass[../main/report.tex]{subfiles}
\begin{document}

\section{Physical system structure}

\missingfigure{Physical overview}

\end{document}

% Design for Redundancy
\section{Design for Redundancy}

When designing a system the PCB is one of the harder things to debug.
If a wire inside the PCB is wrong, there is not munch that can be done, except buying a new PCB.
This becomes apparant when seeing most projects only have a working PCB on the 3rd of maybe even the 4th try. \todo{Citation needed}
Because of this, a strong philosophy has been used in the design for the PCB, as shown in figure \ref{fig:pcb_philosophy}.

\begin{figure}[H]
    \centering
        \includegraphics[width=0.65\textwidth]{pcb/assets/pcb-philosophy.pdf}
    \caption{Design philosophy for the PCB.}
    \label{fig:pcb_philosophy}
\end{figure}

TO make sure the propability of a working PCB will be as high as possible, all aspects of the PCB has one or more backup plans.
All important wires and unused pins have been put onto headers, which can be rerouted manually.
That way each part can be connected to other parts of the board, or to other sources.
Because of this, the board is not optimized for size, but was rather made to optimize for highest possible chance of success.

Another part of this design is that each vital part of the PCB follows a modular design.
Each isolated part will work on its own, with the help of other components, if the rest of the board fails.
A PCB with a broken MCU, but with a functioning FPGA can be connected together and work as one system none the wiser.


% Main Components
\section{Main Components}


\begin{description}

    \item[MCU (EFM32GG990F512-BGA112)] \hfill \\
    The EFM32 Giant Gecko Microcontroller from former Energy Micro, now Silicon Labs, was chosen as the microcontroller for this project.
    One aspect of the task at hand is energy efficiency and this microcontroller is particular efficient in that manner.
    This is a proven microcontroller with a lot of development boards available to us, so it seemed like a safe choice.

    \item[FPGA (XC6SLX45-2CSG324I)] \hfill \\
    The  FPGA of the Spartan-6 family from Xilinx was chosen as the FPGA.
    This particular FPGA has been used for different tasks on the university before, and the support systems are therefore available to us.
    The version was the one used on the PCB.
    A less powerful version of this one was available for testing on development boards in the lab.


    \item[SRAM] \hfill \\
    Yes yes, very good SRAM \todo{More here}

\end{description}


% Input
\documentclass[../main/report.tex]{subfiles}
\begin{document}

\section{Input}

The main method of communication between the microcontroller and a host PC is by USB.
The USB circuitry is designed with ESD protection in an effort to lessen the risk of failure. 
However, if the usb fail, a serial port(RS-232) has been implemented as a backup solution.
If this also fails, the wires from the serial port is put on headers, which can be used as GPIO pins.

\todo{Consider an image here}

\end{document}


% Output
\documentclass[../main/report.tex]{subfiles}
\begin{document}

\section{Output}
The main source of output from the FPGA is an HDMI connector.
This is a novel feature this year, as no previous group has tried to implement it before.

Because of this new challenge, a lot of backup schemes were put in place.
The HDMI connector is put on headers, in case the connector fails.
A separate VGA module is connected to the FPGA, in case the HDMI doesn't work and if this fails, a VGA module is also connected to the microcontroller.

\subsection{HDMI implementation}

The general HDMI specification consists of the Transition Minimized Differential Signaling(TMDS), and some additional wires for details regarding the transmitted signal \cite{hdmi-pinout}.
However, we were able to produce a video feed on a screen from an FPGA using only the TMDS wires by cutting up an HDMI cable and using only those wires.
Upon seeing that this was feasible, we decided to make the HDMI hardware with a TMDS connection only, going in to the FPGA. The resulting hardware was then an HDMI type-A receptacle footprint, the HDMI receptacle and a header between it and the FPGA.
The reason for adding the header was that this setup is equivalent to the aforementioned prototype version so that if the HDMI input wouldn't work we could connect the (header)terminated HDMI cable onto the header.

\todo{Consider an image here}

\end{document}


% Power
\documentclass[../main/report.tex]{subfiles}
\begin{document}

\section{Power}

The system is powered by a 5V mini USB.
The core design implementation is that this connection feeds electricity towards two voltage regulator circuits.
These, in turn power up the rest of the machine.
The USB connection has a backup solution in the form of a header through which the 5V line passes immediately after entering the PCB.
This is done so that the incoming electricity can easily be probed for voltage level and also so that in case the USB connection fails, an external power supply can be connected to the board.

\begin{figure}[H]
	\centering
	\includegraphics[width=0.65\textwidth]{../pcb/assets/power.jpg}
	\caption{Picture of physical power circuit}
	\label{fig: power picture}
\end{figure}
\todo{If the picture is to stay, should probably crop it a bit}

\end{document}


% Bus
\documentclass[../main/report.tex]{subfiles}
\begin{document}

\section{Bus}
The bus between the MCU and FPGA is a EBI connection.
EBI stands for Extrenal Bus Interface, which is a parallel adress/data bus connecting multiple external components like SRAM, MCU, GPU etc etc.
\TODO{More Info about the EBI?}
The EFM32GG990 have dedicated pins for the EBI, which can be found in the datasheet to the MCU.
Headers are placed in between the MCU and FPGA in case of failure on either side of the connection, as well as an easy way to check transmitted signals during debugging.

\end{document}


% Clocks
\documentclass[../main/report.tex]{subfiles}
\begin{document}

\section{Clocks}
Both the MCU and the FPGA need a external clock.
The MCU need two crystals, a low frequency crystal at 32.768kHz and a high frequencey crystal at 48MHz. 
The FPGA, unlike the MCU, need a oscillator. The frequency of the oscillator depends on the requirement of the architecture.

Headers are placed between the oscillator and the FPGA. 
The headers serve as a backup in case an external oscillator is needed and serves as a debugging tool.
The microcontroller clocks on the other hand have a backup solution in that the microcontroller has internal RC-oscillators for use in case the crystals malfunction.

\end{document}


% HDMI?
\documentclass[../main/report.tex]{subfiles}
\begin{document}
Many logically separate parts of the Design work together to enable the video output.
The part that fetches video-data from the current framebuffer in memory.
The part that buffers the incoming data.
The part that generates the video-timings.
And the part that codes the signal as TDMS.
At that comes the clocking-infrastructure.

Note that everything in the section applies equally to DVI-I as HDMI. HDMI is a superset of DVI-I, and nothing exclusive to HDMI is made use of.
Also, as video-timings of DVI are inherited from VGA, everything except the TDMS-encoder would be useful for VGA-output.

The HDMI-specification describes a streaming protocol.
The receiver of the video cannot be relied on to wait and retain a frame when the video-output cannot send.

Thus, central in the design is the buffering-unit.
As the GPU has always first priority on access to the memory, it is not always available to the video-output.
The video-output relies on that there will be enough free timeslots for memory-access when the GPU is otherwise busy to get data for video-out.
Whenever data is opportunistically returned from memory, it is stashed into a FIFO until it's full.

Where data is requested from in memory, is decided by a counter that runs through the address-space of the current framebuffer.
When data is put onto the FIFO, the counter increments. % Yes, `increments` can be intransitive.

When the counter reaches the end of the framebuffer, it is reset to the start the of framebuffer to be displayed next.
This logic also prevents tearing (caused by switching buffers mid-frame).

\end{document}

\section{Footprints}
Predominantly SMDs(surface mounted device) large enough for humans to solder(with a couple of exceptions)

\subfile{../pcb/hdmi.tex}

\todo{...Uncertain ie om at vi endra litt på en del footprints, men pga hvordan fysiske komponenter funker,}

\subsubsection{Prebundled}

\subsubsection{Handmade}
Some footprints we had to make ourselves.
This was done inside Altium.
The specification for the footprints was found in the datasheets of the component in question.

When making the handmade footprints some other thoughts were taken into account.

\begin{itemize}
    \item We need to solder these components
    \item The connection on the components is physical, as long as it leads current, it will work.
    \item Some datasheets didn't match the component exactly, but was for a sister component
\end{itemize}

With this in mind we made footprints which was slightly bigger than it needed to.

\begin{verbatim}
Needed:                 Actual:
--------------          --------------
|            |          |            |
|--|      |--|       |--|--|      |--|--|
|  |      |  |       |  |  |      |  |  |
|..|      |--|       |--|..|      |--|--|
|            |          |            |
--------------          --------------
\end{verbatim}
\end{document}
