\section{PCB}
\label{sec:pcb}

The entire system is mounted on a PCB with wires connecting the different parts.
Designing and soldering this component correctly is therefore extremely important for the system
to work as a whole.

A lot of effort has gone into making sure that if something fails, there is a backup plan.
This means that most of the wires are connected to jumpers, which can be rerouted manually.
Because of this, the board is not optimized for size, but was rather made to optimize for
highest possible chance of success.

This section will explain the overall design, why choices were made and the reason behind these choices.

\subsection{Design Choices}
\label{sec:pcb:design}

\subsubsection{Microcontroller / System Control Unit}
The EFM32 Giant Gecko 32-bit Microcontroller from Energy Micro was chosen as the controller for this project.
A microcontroller from Energy Micro was required for the task and this particular controller is
very energy efficient, which is a plus.
In addition to this, there were a lot of development boards available,
plus over half of the group had experience with this controller from the subject
TDT4258 Energy Efficient Computer Systems.

The EFM32GG990F512-BGA112 was chosen as a powerful enough version of the microcontroller.
Not only did it have enough GPIO pins for our needs, but it could....something %TODO: skriv mer om dette

\subsection{Power Supply}

\subsection{Power Plane}
% TODO: skrive om at vi endra litt på en del footprints, men pga hvordan fysiske komponenter funker,
%       så gjør det ikke noe

\subsection{Footprints}

\subsection{Process}

\subsection{Problems and workarounds}
