\documentclass[../main/report.tex]{subfiles}
\begin{document}

\section{Backup oriented design}

Verifying a PCB design is difficult.
If a wire inside the PCB is broken, there is not much that can be done, except for designing a new PCB.
Because of this, the PCB has been designed with a focus on backup solutions for every design choice and a focus on exposing much of the circuitry on headers.
The reason for this to be able to probe signals for debugging purposes, but also in case manual rewiring is necessary 
\begin{figure}[H]
    \centering
    \includegraphics[width=0.65\textwidth]{../pcb/assets/pcb-overview.pdf}
    \label{fig:pcb-overview}
    \caption{Conceptual overview of the PCB. Green boxes are main solutions. Blue boxes are backup plans.
             Gray boxes labeled "PINS" means there are pins either on the wire itself, or that the box is pins.}
\end{figure}

These backup plans are in place to make sure the board will work, even if some parts are broken.
That way, each component can be connected to other sources than that on the board alone.
Because of this, the board is not optimized for the smallest size possible, but was rather made to optimize for highest possible chance of success.

\end{document}
