\documentclass[../main/report.tex]{subfiles}
\begin{document}
\chapter{Commented tunnel kernel}

\begin{assembly}[label=lst:tunnel-kernel]
; The idea is to draw a green tunnel.
; This effect consists of 2 squares in different sizes and a X across the screen
; To make this happen assume we are on a pixel that should be drawn green
; We then make a series of checks to verify if we are on the correct pixel or not
; If we ever see that we are on the wrong pixel, we will set our mask bit high
; This will make it so that we draw black instead of green
;
; To make this happen we have to use a general purpose register as masking register
; This is because if the masking value gets set high, we don't want to overwrite
; it with low, so we need to remember it.

; First square
ldi $15, 0b111111       ; 63 bitmask
ldi $14, 64             ; Set width & height of screen into $14
and $7, $15, $id_lo     ; Load x value into $7
srl $8, $id_lo, 6       ; Load y value into $8

ldi $data, 0b0000011111100000   ; Set default color to green

mv $12, $0              ; shadow mask - used as temporary mask bit to
                        ; be able to read masking value

ldc $13, 10             ; Set offset from edges
sub $14, $14, $13       ; Set offset from opposite edges

; Draw LEFT line of box
seq $10, $7, $13        ; Checks that we are on correct x value

slt $11, $13, $8        ; Checks that we are inside our box from the top
and $10, $10, $11

slt $11, $8, $14        ; Checks that we are inside our box from the bottom
and $10, $10, $11

or $12, $12, $10        ; Adds result into masking bit
\end{assembly}

\begin{assembly}[label=lst:tunnel-kernel2, firstnumber=36]
; Draw RIGHT line of box
seq $10, $7, $14        ; Same system as above, but checking the other
                        ;side of the box
slt $11, $13, $8
and $10, $10, $11

slt $11, $8, $14
and $10, $10, $11

or $12, $12, $10


; Draw UPPER line of box
seq $10, $8, $13        ; Instead of checking the vertical lines, the horizontal lines
                        ; are checked
slt $11, $13, $7
and $10, $10, $11

slt $11, $7, $14
and $10, $10, $11

or $12, $12, $10


; Draw LOWER line of box
seq $10, $8, $14        ; Same as above, but lower line instead of upper

slt $11, $13, $7
and $10, $10, $11

slt $11, $7, $14
and $10, $10, $11

or $12, $12, $10


; Next square
addi $14, $0, 64        ; Load screen width & height into $14
ldc $13, 11             ; Set new offset from edges
sub $14, $14, $13       ; Set offset from opposite edges

; Draw LEFT line of box
seq $10, $7, $13        ; Same check for the square as previous square

slt $11, $13, $8
and $10, $10, $11

slt $11, $8, $14
and $10, $10, $11

or $12, $12, $10

; Draw RIGHT line of box
seq $10, $7, $14

slt $11, $13, $8
and $10, $10, $11

slt $11, $8, $14
and $10, $10, $11

or $12, $12, $10
\end{assembly}

\begin{assembly}[label=lst:tunnel-kernel3, firstnumber=98]
; Draw UPPER line of box
seq $10, $8, $13

slt $11, $13, $7
and $10, $10, $11

slt $11, $7, $14
and $10, $10, $11

or $12, $12, $10

; Draw LOWER line of box
seq $10, $8, $14

slt $11, $13, $7
and $10, $10, $11

slt $11, $7, $14
and $10, $10, $11

or $12, $12, $10


; Half cross
seq $10, $7, $8     ; if x = y then we are on \

or $12, $12, $10

; Other half cross
addi $13, $0, 0b111111      ; 63 bitmask
sub $14, $13, $8    ; negative y valye, check /

seq $10, $14, $7

or $12, $12, $10

mv $mask, $12       ; Store temporary mask value into actual mask register

? ldi $data, 0x0000 ; Write black if not on either squares or on the cross

ldc $10, 5
add $address_lo, $10, $id_lo
sw                  ; save values
\end{assembly}

\end{document}
