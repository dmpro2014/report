\documentclass[../main/report.tex]{subfiles}
\begin{document}

\chapter{Instruction set architecture}
\label{chap:isa}

This appendix will provide an overview of the instruction set supported by the GPU.

\clearpage

\section{Registers}

The following registers are available:

\begin{table}[H]
    \centering
    \begin{tabular}{|l|l|p{4cm}|l|l|}
    \hline
    \textbf{Number} & \textbf{Name} & \textbf{Description}                                                             & \textbf{R/W}    & \textbf{Size} \\ \hline \hline
    \$0       & zero                & Always contains the value zero                                          & Read-only  & 16   \\ \hline
    \$1, \$2 & id\_hi, id\_lo      & The current thread's ID                                                 & Read-only  & 16   \\ \hline
    \$3, \$4 & address\_hi, address\_lo & Address used by load \& store instructions                              & Read/Write    & 16   \\ \hline
    \$5       & data               & Data loaded/stored by load \& store instructions                        & Read/Write    & 16   \\ \hline
    \$6       & mask               & Conditional instructions will be masked (\autoref{sec:masking}) when this register is set to 1. & Read/Write & 1    \\ \hline
    \$7-\$15  & General-purpose    & ~                                                                       & Read/Write    & 16   \\ \hline
    \end{tabular}
    \caption{Register overview}
    \label{table:registers}
\end{table}

Special registers may be referenced by their name in assembly code.

\clearpage

\section{Predicated instructions}
\label{sec:masking}

The only supported way of conditional execution is through predicated instructions (masking).
All instructions except for \verb/sw/ have a predicated version which will only execute when masking is disabled.

In assembly, an instrution prepended with a question mark will be executed conditionally.
The first bit of the instruction will be set to one for the conditional versions.
These predicated instructions will still be executed, but they will never store their result in the destination register.

Masking is controlled by the dedicated masking register.
Instructions can write to this register to turn masking on and off.
The register is only one bit, and will therefore only keep the least significant bit of data written to it.

\clearpage

\section{Instructions}

\subsection{R-type Instructions}

All R-type instructions have opcode \verb/00000/.
Their syntax, meaning and underlying ALU function are described in table \ref{table:r_type_instructions}.

\begin{table}[H]
    \centering
    \begin{tabular}{llll}
    \textbf{Instruction}   & \textbf{Example}  & \textbf{Meaning}          & \textbf{ALU Function} \\
    \hline
    \hline
    Add                    & add \$1, \$2, \$3 & \$1 = \$2 \verb/+/ \$3    & 0x4          \\
    Subtract               & sub \$1, \$2, \$3 & \$1 = \$2 \verb/-/ \$3    & 0x5          \\
    Multiply               & mul \$1, \$2, \$3 & \$1 = \$2 \verb/*/ \$3    & 0x9          \\ \hline
    And                    & and \$1, \$2, \$3 & \$1 = \$2 \verb/&/ \$3    & 0x6          \\
    Or                     & or \$1, \$2, \$3  & \$1 = \$2 \verb/|/ \$3    & 0x7          \\
    Xor                    & xor \$1, \$2, \$3 & \$1 = \$2 \verb/^/ \$3    & 0x8          \\ \hline
    Set on less than       & slt \$1, \$2, \$3 & \$1 = (\$2 \verb/</ \$3) ? 1 : 0 & 0x3   \\
    Set on equal           & seq \$1, \$2, \$3 & \$1 = (\$2 \verb/==/ \$3) ? 1 : 0 & 0xA  \\ \hline
    Shift Left Logical     & sll \$1, \$2, 10  & \$1 = \$2 \verb/<</ 10    & 0x0          \\
    Shift Right Logical    & srl \$1, \$2, 10  & \$1 = \$2 \verb/>>>/ 10   & 0x1          \\
    Shift Right Arithmetic & sra \$1, \$2, 10  & \$1 = \$2 \verb/>>/ 10    & 0x2          \\
    \end{tabular}
    \caption{R-type instructions}
    \label{table:r_type_instructions}
\end{table}

\begin{table}[H]
    \centering
    \begin{tabular}{|c|c|c|c|c|c|c|c|}
    \multicolumn{1}{c}{1} & \multicolumn{1}{c}{5} & \multicolumn{1}{c}{5}  & \multicolumn{1}{c}{5}  & \multicolumn{1}{c}{5} & \multicolumn{1}{c}{5} & \multicolumn{1}{c}{1} & \multicolumn{1}{c}{5}    \\ \hline
    mask & opcode & rs & rt & rd & sh & X & function \\ \hline
    \end{tabular}
    \caption{Instruction format for R-type instructions.}
    \label{table:r_type_format}
\end{table}

\subsection{I-type Instructions}
\begin{table}[H]
    \centering
    \begin{tabular}{llll}
        \textbf{Instruction} & \textbf{Example} & \textbf{Meaning}          & \textbf{Opcode} \\
        \hline
        \hline
         Load constant       & ldc \$7, 1       & \$7 = constant\_memory[1]  & 0x2 \\
         Add immediate       & addi \$7, \$7, 2 & \$7 = \$7 + 2              & 0x1 \\ \hline
         Load                & lw               & \$data = memory[\$address] & 0x8 \\
         Store               & sw               & memory[\$address] = \$data & 0x4 \\ \hline
         Thread finished     & thread\_finished & Stops executing the kernel & 0x10 \\
         Nop                 & nop              & Do nothing                 & 0x0 \\
    \end{tabular}
    \caption{I-type instructions.}
    \label{table:i_type_instructions}
\end{table}

\begin{table}[H]
    \centering
    \begin{tabular}{|c|c|c|c|c|}
    \multicolumn{1}{c}{1} & \multicolumn{1}{c}{5} & \multicolumn{1}{c}{5}  & \multicolumn{1}{c}{5}  & \multicolumn{1}{c}{16} \\ \hline
    mask & opcode & rs & rd & immediate \\ \hline
    \end{tabular}
    \caption{Instruction format for I-type instructions.}
    \label{table:i_type_format}
\end{table}

\subsection{Pseudo Instructions}

Some additional instructions are supported by the assembler in order to make programming for the GPU easier.
An overview of these instructions is provided in table \ref{table:pseudo_instructions}.

\begin{table}[H]
    \centering
    \begin{tabular}{lll}
        \textbf{Instruction} & \textbf{Example} & \textbf{Translated instruction} \\
        \hline
        \hline
         Move                & mv \$1, \$2      & add \$1, \$0, \$2               \\
         Load immediate      & ldi \$1, 1       & addi \$1, \$0, 1                \\
    \end{tabular}
    \caption{Pseudo instructions}
    \label{table:pseudo_instructions}
\end{table}


\end{document}
