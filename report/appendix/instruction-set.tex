\documentclass[../main/report.tex]{subfiles}
\begin{document}

\chapter{Instruction set architecture}

This appendix will provide an overview of the instruction set supported by the GPU.

\section{Registers}

The following registers are available:

\begin{table}[H]
    \begin{tabular}{|l|l|p{4cm}|l|l|}
    \hline
    \textbf{Number} & \textbf{Name} & \textbf{Description}                                                             & \textbf{R/W}    & \textbf{Size} \\ \hline \hline
    \$0       & zero                & Always contains the value zero                                          & Read-only  & 16   \\ \hline
    \$1 - \$2 & id\_hi, id\_lo      & The current thread's ID                                                 & Read-only  & 16   \\ \hline
    \$3 - \$4 & address\_hi, address\_lo & Address used by load \& store instructions                              & General    & 16   \\ \hline
    \$5       & data               & Data loaded/stored by load \& store instructions                        & General    & 16   \\ \hline
    \$6       & mask               & Conditional instructions will be masked (\autoref{sec:masking}) when this register is set to 1. & Write-only & 1    \\ \hline
    \$7-\$15  & General-purpose    & ~                                                                       & General    & 16   \\ \hline
    \end{tabular}
    \label{table:registers}
    \caption{Register overview}
\end{table}

Special registers may be referenced by their name in assembly code.

\section{Masking}
\label{sec:masking}

\todo{Someone should describe masking of instructions.}

\section{Instructions}

\subsection{R-type Instructions}

All R-type instructions have opcode \verb/00000/.

\begin{table}[H]
    \begin{tabular}{llll}
    \textbf{Instruction}   & \textbf{Example}  & \textbf{Meaning}          & \textbf{ALU Function} \\
    \hline
    \hline
    Add                    & add \$1, \$2, \$3 & \$1 = \$2 \verb/+/ \$3    & 0x4          \\
    Subtract               & sub \$1, \$2, \$3 & \$1 = \$2 \verb/-/ \$3    & 0x5          \\ \hline
    And                    & and \$1, \$2, \$3 & \$1 = \$2 \verb/&/ \$3    & 0x6          \\
    Or                     & or \$1, \$2, \$3  & \$1 = \$2 \verb/|/ \$3    & 0x7          \\
    Xor                    & xor \$1, \$2, \$3 & \$1 = \$2 \verb/^/ \$3    & 0x8          \\ \hline
    Set on less than       & slt \$1, \$2, \$3 & \$1 = (\$2 \verb/</ \$3) ? 1 : 0 & 0x3   \\ \hline
    Shift Left Logical     & sll \$1, \$2, 10  & \$1 = \$2 \verb/<</ 10    & 0x0          \\
    Shift Right Logical    & srl \$1, \$2, 10  & \$1 = \$2 \verb/>>>/ 10   & 0x1          \\
    Shift Right Arithmetic & sra \$1, \$2, 10  & \$1 = \$2 \verb/>>/ 10    & 0x2          \\
    \end{tabular}
    \label{table:r_type_instructions}
    \caption{R-type instructions}
\end{table}

\begin{table}[H]
    \centering
    \begin{tabular}{|c|c|c|c|c|c|c|c|}
    \multicolumn{1}{c}{1} & \multicolumn{1}{c}{5} & \multicolumn{1}{c}{5}  & \multicolumn{1}{c}{5}  & \multicolumn{1}{c}{5} & \multicolumn{1}{c}{5} & \multicolumn{1}{c}{1} & \multicolumn{1}{c}{5}    \\ \hline
    mask & opcode & rs & rt & rd & sh & X & function \\ \hline
    \end{tabular}
    \label{table:r_type_format}
    \caption{Instruction format for R-type instructions.}
\end{table}

\subsection{I-type Instructions}
\todo{Is there a better way to structure this section?}
\begin{table}[H]
    \begin{tabular}{llll}
        \textbf{Instruction} & \textbf{Example} & \textbf{Meaning}          & \textbf{Opcode} \\
        \hline
        \hline
         Load constant       & ldc \$7, 1       & \$7 = constant\_memory[1] & 0x2 \\
         Add immediate       & addi \$7, \$7, 2 & \$7 = \$7 + 2             & 0x1
    \end{tabular}
    \label{table:i_type_instructions}
    \caption{I-type instructions.}
\end{table}

\begin{table}[H]
    \centering
    \begin{tabular}{|c|c|c|c|c|}
    \multicolumn{1}{c}{1} & \multicolumn{1}{c}{5} & \multicolumn{1}{c}{5}  & \multicolumn{1}{c}{5}  & \multicolumn{1}{c}{16} \\ \hline
    mask & opcode & rs & rd & immediate \\ \hline
    \end{tabular}
    \label{table:i_type_format}
    \caption{Instruction format for I-type instructions.}
\end{table}

\begin{table}[H]
    \begin{tabular}{llll}
        \textbf{Instruction} & \textbf{Example} & \textbf{Meaning} & \textbf{Opcode} \\
        \hline
        \hline
         Load                & lw               & \$data = memory[\$address] & 0x8 \\
         Store               & sw               & memory[\$address] = \$data & 0x4\\
    \end{tabular}
    \label{table:memory_type_instructions}
    \caption{Memory instructions}
\end{table}

\begin{table}[H]
    \begin{tabular}{llll}
        \textbf{Instruction} & \textbf{Example} & \textbf{Meaning} & \textbf{Opcode} \\
        \hline
        \hline
         Thread finished     & thread\_finished & Stops executing the kernel & 0x10 \\
         Nop                 & nop              & Do nothing                 & 0x0 \\
    \end{tabular}
    \label{table:misc_instructions}
    \caption{Other instructions}
\end{table}

\end{document}
