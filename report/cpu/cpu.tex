\documentclass[../main/report.tex]{subfiles}
\begin{document}
\chapter{CPU}

\subfile{../cpu/responsibilities.tex}

\section{Communication with Host Computer}
TBD. Might not be needed if this does not end up being an important part of the architecture.

\subfile{../cpu/bus.tex}

\section{Program Initialization}
At the beginning of a program, the CPU must load all kernels it's going to use into the instruction memory of the GPU. This ensures fast execution of kernels, by reducing overhead of a kernel call to a minimum.
The CPU maintains a list of start addresses for different kernels, and uses them to tell the GPU where to start executing.

\section{Running a Program}
Big parts of a demo program is serial in nature, and will not benefit from running on the parallel architecture of the GPU.
These parts of the demo are executed sequentially on the CPU.
When the CPU encounters a part of the program which has a dedicated kernel,
it loads any required parameters into the GPUs constant memory, and then sends a start signal to the GPU.
When the GPU has completed the kernel, it triggers an interrupt on a dedicated wire,
so the CPU can continue executing.

\end{document}