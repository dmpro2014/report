\chapter{CPU}

\section{Responsibilites}

The EFM32GG Microcontroller acts as the CPU in the system.
It is connected to every I/O module except for the graphics output from the GPU.
A program which is stored in the flash memory on the CPU will start running when the board is powered up.
Since the microcontroller has a CPU role, parts of the demo (those which are not parallelizable)
will be executing on this component.
In certain areas of the program, it can choose to execute a "kernel", which will run on the GPU.

\subsubsection{Functionality}

\begin{itemize}
  \item Load all kernels into instruction memory on the GPU.
  \item Start executing program.
  \item When a kernel is encountered any parameters are loaded to the constant storage.
  \item Any data which will not fit in constant storage is loaded to memory.
  \item When all data is ready, a kernel start signal is sent.
  \item Upon the completion of a kernel, the fpga sends a signal back to the CPU.
\end{itemize}


\section{Communication with Host Computer}
TBD. Might not be needed if this does not end up being an important part of the architecture.

\documentclass[../main/report.tex]{subfiles}
\begin{document}

\section{Bus}
The bus between MCU and FPGA is a standard EBI connection.
EBI stands for Extrenal Bus Interface, which is a parallel adress/data bus connecting multiple external components like SRAM, MCU, GPU etc etc.
The MCU have specific pins mapped for this protocol. /TODO{More Info about the EBI?}
Headers are placed in between the MCU and FPGA in case of failure on either side of the connection, as well as an easy way to check transmitted signals during debugging.

\end{document}


\section{Program Initialization}
At the beginning of a program, the CPU must load all kernels it's going to use into the instruction memory of the GPU. This ensures fast execution of kernels, by reducing overhead of a kernel call to a minimum.
The CPU maintains a list of start addresses for different kernels, and uses them to tell the GPU where to start executing.

\section{Running a Program}
Big parts of a demo program or serial in nature, and will not benefit from running on the parallel architecture of the GPU.
This parts are executed sequentially on the CPU.
When the CPU encounters a part of the program which has a dedicated kernel,
it loads any required parameters into the GPUs constant memory, and then sends a start signal to the GPU.
When the GPU has completed the kernel, it triggers an interrupt on a dedicated wire,
so the CPU can continue executing.
