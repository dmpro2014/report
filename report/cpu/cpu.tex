\documentclass[../main/report.tex]{subfiles}
\begin{document}
\chapter{CPU}

Now that we have seen how to write a kernel that colors the screen with a beautiful green color, 
it is time to see what is actually happening on the Demolicious System during this execution. 

Our journey starts in the main control unit of the Demolicious computer, namely the CPU, which is implemented on an EFM32GG Microcontroller.

In this chapter we will follow a kernel from load to execution and explain what happens behind the scenes on the CPU, which is the component on Demolicious that runs the C code seen in the previous chapter.

\subfile{../cpu/functionality.tex}

\subfile{../cpu/bus.tex}

\subfile{../cpu/load-kernel.tex}

\subfile{../cpu/run-kernel.tex}

\subfile{../cpu/load-constant.tex}

\subsection{Summary}

We have now seen how our kernel is bootstrapped and executed.
The assembled kernel is uploaded to the GPU which is then given a command to run said kernel with one thread per pixel.
Kernels can also read parameters which the host program can easily vary between kernel runs.

\end{document}
