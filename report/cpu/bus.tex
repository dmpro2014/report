\documentclass[../main/report.tex]{subfiles}
\begin{document}
\section{Communication with the GPU}

The backbone of the interaction between the CPU and the GPU is the wide bus connecting them.
Since the GPU is designed to increase the performance of the system by parallelizing operations,
it's important that the bus does not introduce too much overhead.
We therefore designed a parallel bus based on the EBI (External Bus Interface) specifications provided
in the reference manual for the EFM32GG\cite[p.175]{efm32gg}.
The bus has a 16-bit data line, and a 20-bit address line, as well as six control signals.

All the pins connected between the MCU and the FPGA follow the specifications from the EFM32GG's Reference Manual\cite[p.175]{efm32gg}.
This allows us to utilize the built-in support for EBI which memory maps the whole bus on the microcontroller, and take advantage of utility functions in the \emph{emlib} software library\cite{emlib}.
The mapping of addresses can be seen in figure \ref{fig:memory_map}.

The CPU can interface three different memories for different purposes:

\begin{enumerate}
    \item Constant memory, for storing constants which may be read by kernels.
    \item Instruction memory, for storing the kernels.
    \item Data memory, for storing the framebuffer, in addition to other data for kernels.
\end{enumerate}

\begin{figure}[H]
    \centering
    \begin{tabularx}{\textwidth}{|X|X|X|X|X|}
    \multicolumn{1}{c}{0x80000000} & \multicolumn{1}{c}{0x80040000} & \multicolumn{1}{c}{0x80080000} & \multicolumn{1}{c}{0x80100000} & \multicolumn{1}{c}{0x84000000} \\ \hline
    Constant Memory & Instruction Memory & Kernel start area & X & External memory \\ \hline
    \end{tabularx}
    \caption{Overview of memory mapped address spaces on the CPU.}
    \label{fig:memory_map}
\end{figure}

Writing to all of these memories is done by writing to the memory addresses
outlined in figure \ref{fig:memory_map},
in the same way as writing data locally on the microcontroller.
Data can also be read back from the external GPU memory.
Even the task of starting kernels is done the same way, to be able to utilize the same bus for all GPU communication.

\end{document}
