\documentclass[../main/report.tex]{subfiles}
\begin{document}
Many logically separate parts of the Design work together to enable the video output.
The part that fetches video-data from the current framebuffer in memory.
The part that buffers the incoming data.
The part that generates the video-timings.
And the part that codes the signal as TDMS.
At that comes the clocking-infrastructure.

Note that everything in the section applies equally to DVI-I as HDMI. HDMI is a superset of DVI-I, and nothing exclusive to HDMI is made use of.
Also, as video-timings of DVI are inherited from VGA, everything except the TDMS-encoder would be useful for VGA-output.

The HDMI-specification describes a streaming protocol.
The receiver of the video cannot be relied on to wait and retain a frame when the video-output cannot send.

Thus, central in the design is the buffering-unit.
As the GPU has always first priority on access to the memory, it is not always available to the video-output.
The video-output relies on that there will be enough free timeslots for memory-access when the GPU is otherwise busy to get data for video-out.
Whenever data is opportunistically returned from memory, it is stashed into a FIFO until it's full.

Where data is requested from in memory, is decided by a counter that runs through the address-space of the current framebuffer.
When data is put onto the FIFO, the counter increments. % Yes, `increments` can be intransitive.

When the counter reaches the end of the framebuffer, it is reset to the start the of framebuffer to be displayed next.
This logic also prevents tearing (caused by switching buffers mid-frame).

\end{document}