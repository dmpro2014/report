\documentclass[../main/report.tex]{subfiles}
\begin{document}

\chapter{Introduction}
\label{sec:intro}

This report presents the TDT4295 Computer Design Project at NTNU for the fall semester of 2014.

The project is the single task of a double-credits course every fall in which a group of students make a working computer from scratch.
This year's group was made out of 9 students from the Computer Science department.

\section{Assignment}

The Computer Design Project's primary subtasks include designing a custom printed circuitboard (PCB) and  implementing a custom processor architecture on an FPGA.
Together with a microcontroller (MCU) and a choice of I/O components, these will form a complete and working system.
The project is evaluated based on this report and an oral presentation of the work, as well as a prototype demonstration.

The task this year was to create a processor inspired by GPU architectures.
Core requirements included having multiple processor cores and a graphical display output.
Gunnar Tufte and Yaman Umuroglu, served as advisors for the group throughout the semester and assisted in administrative tasks.

\newpage

\subsection{Original Assignment Text}
\todo{Keep assignment text inside "quotation" environment?}

\begin{quotation}
	\subsubsection{Construct a graphics processing unit (GPU) inspired processor}
    \noindent GPUs play a large role in graphical applications as well as high performance computing.
    They are typically constructed around the SIMD (single instruction multiple data) paradigm and
    include special hardware for accelerating graphics-related operation. The idea is to make a
    GPU-inspired processor architecture that exploits the possibility of parallel computation on a
    single chip. The GPU must be a multi-core system.
    
    \subsubsection{Additional requirements}
    \todo{Fix paragraph spacing and indentation}
    \noindent Your processor will be implemented on an FPGA, and you are free to choose how to
    realize your computer architecture. Studying the architecture of general multicore processors
    and parallel machines options can be a good starting point.
    
    Energy efficiency should be a primary consideration in all phases of the project, from early
    design decisions to how software is written.
    
    The task should also include a suitable application that can produce a graphical output on a
    display to demonstrate the processor.
    
    The unit must utilize a Silicon Labs EFT32 series microcontroller (to act as an I/O processor)
    and a Xilinx FPGA (to implement your architecture on). The budget is 10.000 NOK, which must
    cover components and PCB production. The unit design must adhere to the limits set by the
    course staff at any given time.
\end{quotation}
\newpage

\subsection{A Graphics Accelerator}

What is the approach we took to this project?
What did we want to make?
Why did we want to make that?
What did we hope to achieve?

\section{Modern Graphics Processing Units}

What is a GPU?
What is its purpose?

\subsection{Design Considerations}

What issues are important to deal with in designing a GPU?

\subsection{A Parallel Universe}

Such parallelism. Works for the things it does, you know.

\subsection{Bottlenecked By Memory}

Much memory. Wow.

\subsection{The Big Actors}

Who makes graphics accelerators?
For what market do they make these?
What is performance like for?

\section{Demolicious}

What approach did we take in making a graphics accelerator?
What trade-offs did we have to make?
What were our concerns?
What kind of numbers did we hope to achieve?

The project will enter history like a wrecking ball.

Its name is Licious. Demo Licious.

\subsection{Solution Requirements}

How did we arrive at our list of requirements?

\begin{table}[htp]
    \centering
    \begin{tabular}{|l|p{8cm}|l|}
        \hline
        \textbf{Identifier}           & \textbf{Description}                & \textbf{Priority} \\ \hline
        GOAL1  & Computer should display graphics on screen                           & HIGH    \\ \hline
        GOAL2  & Computer should be general purpose                                   & HIGH    \\ \hline
        GOAL3  & Computer should drive video from FPGA                                & MEDIUM  \\ \hline
        GOAL4  & Frame rate should be around 30 fps                                   & MEDIUM  \\ \hline
        GOAL5  & Computer should use HDMI to display the graphics                     & MEDIUM  \\ \hline
        GOAL6  & Computer should have an example application in form of a demo        & LOW     \\ \hline
        GOAL7  & Computer should have a toolchain to make life easier for programmers & LOW     \\ \hline
    \end{tabular}
    \caption{Goals set for the computer}
    \label{tab:goals}
\end{table}

\newpage
\section{Structure of the Report}

To be announced

\end{document}