\documentclass[../main/report.tex]{subfiles}
\begin{document}

\chapter{Introduction}
\label{sec:intro}

This report presents the TDT4295 Computer Design Project at NTNU for the fall semester of 2014.

The project is the single task of a course every fall in which a group of students make a working computer from scratch.
This year's group was made out of 9 students from the Computer Science department.
Gunnar Tufte and Yaman Umuroglu, served as advisors for the group throughout the semester and assisted in administrative tasks.

\section{Assignment}

The Computer Design Project's primary tasks include making a custom printed circuitboard (PCB) and  implementing a custom processor architecture on an FPGA.
Together with a microcontroller (MCU) and a choice of I/O components, these will form a complete and working system.
The project is evaluated based on this report and an oral presentation of the work, as well as a prototype demonstration.

The task this year was to create a processor inspired by GPU architectures.
Core requirements included having multiple processor cores and a graphical display output.

\newpage

\subsection{Original Assignment Text}
\todo{Keep assignment text inside "quotation" environment?}

\begin{quotation}
	\subsubsection{Construct a graphics processing unit (GPU) inspired processor}
    \noindent GPUs play a large role in graphical applications as well as high performance computing.
    They are typically constructed around the SIMD (single instruction multiple data) paradigm and
    include special hardware for accelerating graphics-related operation. The idea is to make a
    GPU-inspired processor architecture that exploits the possibility of parallel computation on a
    single chip. The GPU must be a multi-core system.
    
    \subsubsection{Additional requirements}
    \noindent Your processor will be implemented on an FPGA, and you are free to choose how to
    realize your computer architecture. Studying the architecture of general multicore processors
    and parallel machines options can be a good starting point.\\
    
    \noindent Energy efficiency should be a primary consideration in all phases of the project, from early
    design decisions to how software is written.\\
    
    \noindent The task should also include a suitable application that can produce a graphical output on a
    display to demonstrate the processor.\\
    
    \noindent The unit must utilize a Silicon Labs EFM32 series microcontroller (to act as an I/O processor)
    and a Xilinx FPGA (to implement your architecture on). The budget is 10.000 NOK, which must
    cover components and PCB production. The unit design must adhere to the limits set by the
    course staff at any given time.
\end{quotation}
\newpage

\subsection{A Graphics Accelerator}

\textit{What is the approach we took to this assignment?}
\textit{What did we want to make?}
\textit{Why did we want to make that?}
\textit{What did we hope to achieve?}

The group decided to make the custom GPU inspired processor as an accelerator processor.
This means that it would not be designed to run entire programs on its own, but would rather handle particular parallelizable parts.
In this case, the program parts that would be offloaded to the accelerator would be graphics related operations.
It was also decided to send graphical output to screen directly from the accelerator, akin to modern GPUs.
This would form a graphical computer system similar to how modern PCs are organized.

\section{Modern Graphics Processing Units}

\textit{What is a GPU?}
\textit{What is its purpose?}

\todo{Fix source references, "Computer Organization and Design", Appendix C}

Modern GPUs are, in a way, an evolution of former Video Graphics Array (VGA) controllers.
A VGA controller of the early 1990s served as a memory controller and display generator that wrote framebuffer values to a display.
As technology advanced, it received hardware to perform specific graphics related functions.
This eventually evolved into a processor, with its own memory, that incorporated a full set of graphical functions.

A GPU's primary purpose has traditionally been to offload graphical calculations from the CPU and render graphical data to a screen.
Graphical functions are accessed through graphical APIs like DirectX\textsuperscript{TM} and OpenGL\textsuperscript{TM}.
Today, GPUs also have general computing capabilities and may serve as co-processors for the CPU in addition to handling their graphical duties.
Non-graphics applications for a GPU include image processing, video encoding, and many scientific computing problems and other large, highly regular calculations.

\todo{Not really true anymore. Google unified shading architecture} 
GPUs usually have specialized hardware units for accelerating tasks like vertex shading and pixel shading.
In general, at least one GPU is present in every non-server computer now.
These can take the form of discrete chips or be integrated with the CPU.

\subsection{The GPU Market}

\textit{Who makes graphics accelerators?}
\textit{For what market do they make these?}
\textit{What is performance like for modern GPUs?}

\todo{fix references here}
% PC market reference:
% http://www.zdnet.com/latest-retail-figures-show-a-stable-pc-market-led-by-apple-and-cheap-notebooks-7000034024/
% NDP provide consumer tracking services. Link: https://www.npd.com/wps/portal/npd/us/about-npd/consumer-panel/
% This is their report: https://www.npd.com/wps/portal/npd/us/news/press-releases/apple-and-chrome-boost-us-retail-pc-back-to-school-sales/


In general, at least one GPU is present in every PC these days, and the market for PCs has been relatively stable for many years.
These GPUs can take the form of discrete chips or be integrated with the CPU.
Intel, AMD, and NVIDIA have market shares for PC GPUs.
Intel is largest overall but only makes GPUs integrated with their CPUs.
AMD and NVIDIA share the discrete GPU domain.
A big market for the more powerful discrete GPUs is the computer gaming industry.
This market has contributed a great deal to the rapid progression of graphics technologies.
With the advent of general purpose GPUs, scientific computing also have an interest in powerful GPUs.

In the past 6 years, the booming market for mobile smart devices has introduced a new arena for GPUs.
Mobile devices are currently outselling workstations, and every new mobile device sold today is shipped with a small-format GPU.
These GPUs are generally integrated in the device's system-on-a-chip (SoC), and have slightly different design considerations than traditional workstation GPUs.
As for everything else that runs on batteries, power consumption is a primary concern in this format.
Traditionally, this is a concern that GPU design has not needed to prioritize.
The mobile GPU format has opened up for other GPU design companies than the three mentioned above.
The three dominant GPU design companies in the SoC market are Qualcomm, ARM, and Imagination Technologies.

% PC GPU market share reference:
% http://jonpeddie.com/news/comments/gpu-shipments-marketwatch-q2-2014-charts-and-images/
% and: http://jonpeddie.com/publications/add-in-board-report/
% SoC market share reference: http://jonpeddie.com/press-releases/details/qualcomm-single-largest-proprietary-gpu-supplier-imagination-technologies-t/
% "Jon Peddie Research (JPR) is a technically oriented computer graphics marketing and management-consulting firm"

\subsection{An Enormous Task}

Producing computer graphics is a highly processing intensive task.
To illustrate why computers and mobile devices include a separate GPU, lets look at some back-of-the-envelope calculations.

Consumers expect their computers to display videos and games in Full HD resolution (1920*1080 pixels), which adds up to about 2 million pixels per frame.
To color one pixel accurately in a 3D environment, the processor typically needs to calculate vectors in 3D space, interpolate texture data, adjust for light intensity, and more.
All of these tasks require several floating point operations.
For an optimistic estimate in a fairly simple scene, assume that the CPU needs to perform 1000 operations per pixel. \todo{fix some numbers}
In this case, even with a clock speed of 4 GHz and no other interfering work, a Full HD screen will receive an unacceptably low framerate.
Fortunately, pixels can often be computed independently of each other.
CPUs, however, are optimized for single-thread performance and fail to take advantage of this.

\subsection{A Parallel Universe}

For a CPU, latency is of primary concern.
The next instruction often depends on the result of the previous one.
For that reason the CPU completes a single instruction as fast as possible.

This makes the CPU very good at problems with a low level of parallelism.
Latency becomes less of a concern, when problems get more parallel.
GPUs are optimized for problems with a high degree of parallelism.

\todo{Another main factor to the difference is that CPUs have to dedicate a big part of the chip for the L2 Cache to reduce memory latency, while a GPU uses much more of the chip to actual ALU logic. Talk about this?}
Instead of prioritizing latency, GPUs focus on throughput.
The fast single threaded cores in CPUs are replaced with a large quantity of smaller, slower cores.
The cores in the GPU are slow compared to the CPU, giving them high latency, but can execute a lot of instructions simultaneously giving the GPU a high amount of throughput.
To achieve the massive throughput required in modern graphics the GPU has to become massively parallel 
GPUs features deep pipelines allowing allowing a multitude of instructions to be executed in parallel.
The processing cores contain multiple execution units, allowing multiple threads to share the resources.
Sophisticated dynamic scheduling schemes works to maximize the utilization of GPU resources.

\subsection{At The Mercy of Memory}

%"Modern GPUs are highly parallel, as shown in Figure C.2.5. For example, the GeForce 8800 can process 32 pixels per clock, at 600 MHz. Each pixel typically requires a color read and write and a depth read and write of a 4-byte pixel. Usually an average of two or three texels of four bytes each are read to generate the pixel’s color. So for a typical case, there is a demand of 28 bytes times 32 pixels = 896 bytes per clock. Clearly the bandwidth demand on the memory system is enormous."

\textit{Some numbers. Graphics workloads demand very high data tranfer rates.}

The GPU can deliver very high throughput because of the massive amount of processing units, but with great computing power comes great memory demand.
The challenge in GPUs is keeping the computational units supplied with enough data.
\todo{Insert number}
To supply the processing units with data, the GPU has to have a complex memory system. 
The memory system uses techniques such as batching memory requests, compression, and different types of memories.
The memory is organized in a hierarchy of memories and caches, of different sizes.
The dram interface for the GeForce 8800 had 384 pins, making a bandwidth of up to 96 GB/s possible.
To further increase the bandwidth compression is used to reduce the amount of data that has to be transmitted.
To increase the memory throughput memory requests are saved up and sent in batches.
This creates a lot of latency for memory requests, which means hiding this latency from the programmer is important.
 
\begin{itemize}
	\item The memory buses are wide.
\end{itemize}


\subsection{Design Considerations}

\textit{What kind of functional requirements does a GPU have?}
\textit{What issues are important to deal with when designing a GPU?}
- keeping the memory bus saturated
- scheduling, to utilize the resources
- thread divergence(branching)

\subsection{NVIDIA's Tesla Architecture} % or whatever architecture you want...



\emph{Unified Shader Architecture}
\todo{Move this paragraph up?}
In 2006 NVIDIA released their first GPUs with a so called \emph{unified shader architecture}, named Tesla.
Before this GPUs had seperate and dedicated hardware for the main graphics operations, 
but with this architecture the major operations are executed on the same hardware.
This is significantly different from earlier GPU microarchitectures, and opened
for the possibility to do other calculations on a GPU than graphics processing.

To open the powers of the graphics card for other applications, there was a need for a new API that could be
used run code on the GPUs. It has been possible to do non-graphical calculations on GPUs before by 
redifining problems as graphical problems, but this is highly impractical. 
So with the new architecture NVIDIA released a framework for running code on their GPUs called CUDA, 
which let's the programmer run arbitrary code that gains he benefit of the massively parallel hardware.


\emph{General numbers of cores etc}


\emph{Streaming Multiprocessors}



\emph{Streaming Processors}

\emph{Special Function Unit}

\emph{Shared memory}

\emph{Warps}

\emph{Blocks / memory etc? Maybe not}



\subsection{CUDA/OpenCL}
\emph{Introduce the CUDA programming model. Walk through a simple example storing values to memory.}

\emph{Introductory paragraph explaining what CUDA and OpenCL is}

\emph{Explain how kernels are are written, uploaded and data moved}

\emph{Give a greenscreen example}

\emph{Maybe give a scientific example with result read back. But only if we show it for Demolicious as well}


\section{Demolicious}

\textit{What approach did we take in making a graphics accelerator?}
\textit{What trade-offs did we have to make?}
\textit{What were our concerns?}
\textit{What kind of numbers did we hope to achieve?}


The Demolicious design is inspired by modern GPU design, like the one described above.
Modern GPUs are very complex, and have long development cycles.
For the Demolicious system the design had to be greatly simplified.
Because of both time and space constraints, many features that define modern GPUs had to be left out.
There is no branching, and there are no caches because of this.
Modern GPUs have dynamic scheduling to better utilize the resources.
The Demolicious GPU uses barrel processing as a static scheduling scheme, and to hide memory latency.



The project will enter history like a reking ball.

Its name is Licious. Demo Licious.

\subsection{Solution Requirements}

\textit{How did we arrive at our list of requirements?}

\begin{table}[htp]
    \centering
    \begin{tabular}{|l|p{8cm}|l|}
        \hline
        \textbf{Identifier}           & \textbf{Description}                & \textbf{Priority} \\ \hline
        GOAL1  & Computer should display graphics on screen                           & HIGH    \\ \hline
        GOAL2  & Computer should be general purpose                                   & HIGH    \\ \hline
        GOAL3  & Computer should drive video from FPGA                                & MEDIUM  \\ \hline
        GOAL4  & Frame rate should be around 30 fps                                   & MEDIUM  \\ \hline
        GOAL5  & Computer should use HDMI to display the graphics                     & MEDIUM  \\ \hline
        GOAL6  & Computer should have an example application in form of a demo        & LOW     \\ \hline
        GOAL7  & Computer should have a toolchain to make life easier for programmers & LOW     \\ \hline
    \end{tabular}
    \caption{Goals set for the computer}
    \label{tab:goals}
\end{table}

\newpage
\section{Structure of the Report}

To be announced

\end{document}