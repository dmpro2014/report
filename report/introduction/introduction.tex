\documentclass[../main/report.tex]{subfiles}
\begin{document}

\chapter{Introduction}
\label{sec:intro}

This report presents the TDT4295 Computer Design Project at NTNU for the fall semester of 2014.

The project is the single task of a course every fall in which a group of students make a working computer from scratch.
This year's group was made out of 9 students from the Computer Science department.
Gunnar Tufte and Yaman Umuroglu, served as advisors for the group throughout the semester and assisted in administrative tasks.

\section{Assignment}

The Computer Design Project's primary tasks include making a custom printed circuitboard (PCB) and  implementing a custom processor architecture on an FPGA.
Together with a microcontroller (MCU) and a choice of I/O components, these will form a complete and working system.
The project is evaluated based on this report and an oral presentation of the work, as well as a prototype demonstration.

The task this year was to create a processor inspired by GPU architectures.
Core requirements included having multiple processor cores and a graphical display output.

\newpage

\subsection{Original Assignment Text}
\todo{Keep assignment text inside "quotation" environment?}

\begin{quotation}
	\subsubsection{Construct a graphics processing unit (GPU) inspired processor}
    \noindent GPUs play a large role in graphical applications as well as high performance computing.
    They are typically constructed around the SIMD (single instruction multiple data) paradigm and
    include special hardware for accelerating graphics-related operation. The idea is to make a
    GPU-inspired processor architecture that exploits the possibility of parallel computation on a
    single chip. The GPU must be a multi-core system.
    
    \subsubsection{Additional requirements}
    \noindent Your processor will be implemented on an FPGA, and you are free to choose how to
    realize your computer architecture. Studying the architecture of general multicore processors
    and parallel machines options can be a good starting point.\\
    
    \noindent Energy efficiency should be a primary consideration in all phases of the project, from early
    design decisions to how software is written.\\
    
    \noindent The task should also include a suitable application that can produce a graphical output on a
    display to demonstrate the processor.\\
    
    \noindent The unit must utilize a Silicon Labs EFT32 series microcontroller (to act as an I/O processor)
    and a Xilinx FPGA (to implement your architecture on). The budget is 10.000 NOK, which must
    cover components and PCB production. The unit design must adhere to the limits set by the
    course staff at any given time.
\end{quotation}
\newpage

\subsection{A Graphics Accelerator}

\textit{What is the approach we took to this assignment?}
\textit{What did we want to make?}
\textit{Why did we want to make that?}
\textit{What did we hope to achieve?}

The group decided to make the custom GPU inspired processor as an accelerator processor.
This means that it would not be designed to run entire programs on its own, but would rather handle particular parallelizable parts.
In this case, the program parts that would be offloaded to the accelerator would be graphics related operations.
It was also decided to send graphical output to screen directly from the accelerator, akin to modern GPUs.
This would form a graphical computer system similar to how modern PCs are organized.

\section{Modern Graphics Processing Units}

\textit{What is a GPU?}
\textit{What is its purpose?}

\todo{Fix source references, "Computer Organization and Design", Appendix C}

Modern GPUs are, in a way, an evolution of former Video Graphics Array (VGA) controllers.
A VGA controller of the early 1990s served as a memory controller and display generator that wrote framebuffer values to a display.
As technology advanced, it received hardware to perform specific graphics related functions.
This eventually evolved into a processor, with its own memory, that incorporated a full set of graphical functions.

A GPU's primary purpose has traditionally been to offload graphical calculations from the CPU and render graphical data to a screen.
Graphical functions are accessed through graphical APIs like DirectX\textsuperscript{TM} and OpenGL\textsuperscript{TM}.
Today, GPUs also have general computing capabilities and may serve as co-processors for the CPU in addition to handling their graphical duties.
Non-graphics applications for a GPU include image processing, video encoding, and many scientific computing problems and other large, highly regular calculations.

GPUs usually have specialized hardware units for accelerating tasks like vertex shading and pixel shading.

\subsection{The Big Actors}

\textit{Who makes graphics accelerators?}
\textit{For what market do they make these?}
\textit{What is performance like for modern GPUs?}

\subsection{Graphics Calculations are Demanding}

Graphics is a highly processing intensive task. 
For each pixel on the screen the processor has to calculate the colour of the pixel, taking factors such as light, shadows and texture into account.
Consumers expect that their computer can display videos, and games with HD (1920*1080 pixels) resolution at 60 frames per second.
This means that the computer has to process about 12.5 billion pixels per second.
Assuming a processor running at a frequency of 4 GHz, and that it can process 1 pixel per cycle.
This processor requires about three seconds to render one second.
The CPU is optimized for high single thread performance, and fails to take advantage of the data parallelism in graphics.

\subsection{A Parallel Universe}

GPUs are optimized for problems with a high degree of parallelism.
They replace the fast single threaded cores in CPUs with a large quantity of smaller slower cores.
For CPUs latency is a primary concern, the faster it can complete the instruction, the faster it can begin executing the next.
GPUs maximises the work done by prioritising throughput.
The cores in the GPU are slow compared to the CPU, giving them a high latency, but can execute a lot of instructions simultaneously giving the GPU a high amount of throughput. 

And with great computing power comes great memory demand.

\subsection{At The Mercy of Memory}
"
Modern GPUs are highly parallel, as shown in Figure C.2.5. For example, the GeForce 8800 can process 32 pixels per clock, at 600 MHz. Each pixel typically requires a color read and write and a depth read and write of a 4-byte pixel. Usually an average of two or three texels of four bytes each are read to generate the pixel’s color. So for a typical case, there is a demand of 28 bytes times 32 pixels = 896 bytes per clock. Clearly the bandwidth demand on the memory system is enormous.
"

\textit{Some numbers. Graphics workloads demand very high data tranfer rates.}

\subsection{Design Considerations}

\textit{What kind of functional requirements does a GPU have?}
\textit{What issues are important to deal with when designing a GPU?}

\subsection{nVidia's Kepler Architecture} % or whatever architecture you want...

\textit{Explain nVidia!}

\section{Demolicious}

\textit{What approach did we take in making a graphics accelerator?}
\textit{What trade-offs did we have to make?}
\textit{What were our concerns?}
\textit{What kind of numbers did we hope to achieve?}

The project will enter history like a wrecking ball.

Its name is Licious. Demo Licious.

\subsection{Solution Requirements}

\textit{How did we arrive at our list of requirements?}

\begin{table}[htp]
    \centering
    \begin{tabular}{|l|p{8cm}|l|}
        \hline
        \textbf{Identifier}           & \textbf{Description}                & \textbf{Priority} \\ \hline
        GOAL1  & Computer should display graphics on screen                           & HIGH    \\ \hline
        GOAL2  & Computer should be general purpose                                   & HIGH    \\ \hline
        GOAL3  & Computer should drive video from FPGA                                & MEDIUM  \\ \hline
        GOAL4  & Frame rate should be around 30 fps                                   & MEDIUM  \\ \hline
        GOAL5  & Computer should use HDMI to display the graphics                     & MEDIUM  \\ \hline
        GOAL6  & Computer should have an example application in form of a demo        & LOW     \\ \hline
        GOAL7  & Computer should have a toolchain to make life easier for programmers & LOW     \\ \hline
    \end{tabular}
    \caption{Goals set for the computer}
    \label{tab:goals}
\end{table}

\newpage
\section{Structure of the Report}

To be announced

\end{document}