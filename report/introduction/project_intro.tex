\documentclass[../main/report.tex]{subfiles}
\begin{document}

\chapter{Computer Design Project}
\label{sec:intro}

This report presents the TDT4295 Computer Design Project at NTNU for the fall semester of 2014.

The course is held every fall, and consists of a single task in which a group of students make a working computer from scratch.
This report's group was made out of 9 students from the Computer Science department.
Gunnar Tufte and Yaman Umuroğlu served as advisors for the group throughout the semester and assisted in administrative tasks.

\section{Assignment}

The Computer Design Project's primary tasks included making a custom printed circuitboard (PCB) and implementing a custom processor architecture on an FPGA.
Together with a microcontroller (MCU) and a choice of I/O components, these were to form a complete and working system.
The project was evaluated based on this report and an oral presentation of the work, as well as a prototype demonstration.

The specific assignment for this year was to create a processor inspired by GPU architectures.
Core requirements included having multiple processor cores and a graphical display output.

\newpage

\vspace*{1cm}
\subsection{Original Assignment Text}

\vspace{0.5cm}
\noindent

\begin{quotation}

	\subsubsection{Construct a graphics processing unit (GPU) inspired processor}
    \noindent GPUs play a large role in graphical applications as well as high performance computing.
    They are typically constructed around the SIMD (single instruction multiple data) paradigm and
    include special hardware for accelerating graphics-related operation. The idea is to make a
    GPU-inspired processor architecture that exploits the possibility of parallel computation on a
    single chip. The GPU must be a multi-core system.
    
    \subsubsection{Additional requirements}
    \noindent Your processor will be implemented on an FPGA, and you are free to choose how to
    realize your computer architecture. Studying the architecture of general multi-core processors
    and parallel machines options can be a good starting point.\\
    
    \noindent Energy efficiency should be a primary consideration in all phases of the project, from early
    design decisions to how software is written.\\
    
    \noindent The task should also include a suitable application that can produce a graphical output on a
    display to demonstrate the processor.\\
    
    \noindent The unit must utilize a Silicon Labs EFM32 series microcontroller (to act as an I/O processor)
    and a Xilinx FPGA (to implement your architecture on). The budget is 10 000 NOK, which must
    cover components and PCB production. The unit design must adhere to the limits set by the
    course staff at any given time.
\end{quotation}
\newpage

\section{A Parallel Processing Accelerator}

The group decided to make the custom GPU inspired processor as an accelerator processor for parallelizable computations.
This means that it would not be designed to run entire programs on its own, but would instead be tasked with particular parallelizable parts of a program.
Its purpose would primarily be to run graphics related operations.
However, it would be designed to be programmable with general arithmetic and logic operations, and not have specialized hardware units for performing graphics operations.
The group was determined to also send graphics output directly from the accelerator to screen, as is common for a GPU.

The accelerator design needed be implemented on an FPGA.
The FPGA was to be mounted on a PCB together with a microcontroller, memory, and HDMI port, and form a graphical computer system similar to how modern PCs are organized.


\section{Structure of the Report}

The accelerator processor will fill the role of a modern GPU in the system presented by this report.
Therefore, it will be referred to in this report as the GPU.

Having introduced the problem, the report will continue by giving a short introduction to what a GPU is and what they are for.
It will present concepts and challenges in graphical computations and GPU design.
These topics help to appreciate the need for a GPU in modern computers, and to understand the trade-offs and optimizations involved in its design.

The system created in this project, and its purpose, will be introduced in light of modern GPU concepts.
In the solution part, a detailed explanation of the system will be given by roughly following the path of an executing program, going from high to low levels of abstraction.

The physical product produced in this project, and the test and verification methods involved in making it work, will also be presented.
The results chapter will go through which parts of the system worked and which did not, and will look at measurements of its performance.
Lastly, the discussion chapter will comment on some of the difficulties with the completed system, and on topics such as possible further work.

Note that the reader of this report is expected to have a basic understanding of logic design, computer components, and programming.

\end{document}
