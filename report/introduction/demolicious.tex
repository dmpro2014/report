\documentclass[../main/report.tex]{subfiles}
\begin{document}

\chapter{Demolicious}
\label{sec:demolicious}


% Why is the name of our system "Demolicious"?@
% What approach did we take in making a graphics accelerator?@
% What trade-offs did we have to make?@
% What were our concerns?@
% What kind of numbers did we hope to achieve?@

The system created in this project is named Demolicious.
It is made for running graphical demo's, which is the inspiration for its name.
A graphical demo is a visually pleasing programming feat made to demonstrate the capabilities of the computer as well as the programmer.

Demolicious is inspired by modern PCs' CPU-GPU coordination, both in its programming model and its architectural design.
The CPU will handle programs by default, and offload parallelizable tasks to the GPU.
It will run on a microcontroller.
The GPU will handle parallelizable tasks, like the many graphical operations in a demo, and send graphical data to screen.
Its architecture will be designed and implemented on an FPGA.

Modern GPUs have long development cycles and are very complex.
Demolicious' GPU architecture is necessarily a greatly simplified version.
Because of both time and space constraints, many features that define modern GPUs had to be left out in its design.
The GPU has no branching, and there are no caches.
Modern GPUs have dynamic scheduling to better utilize the resources.
The Demolicious GPU uses barrel processing as a static scheduling scheme, and to hide memory latency.

\section{Solution Requirements}

% How did we arrive at our list of requirements?@

\begin{table}[htp]
    \centering
    \begin{tabular}{|p{8cm}|l|}
        \hline
        \textbf{Demolicious Functional Goals}                & \textbf{Priority} \\ \hline
        Demolicious should display graphics on screen                           & HIGH    \\ \hline
        Demolicious should be general purpose                                   & HIGH    \\ \hline
        Demolicious should drive video from GPU module on FPGA                  & MEDIUM  \\ \hline
        Demolicious should handle an output rate of about 30 frames per second  & MEDIUM  \\ \hline
        Demolicious should use HDMI for its graphics output	                    & MEDIUM  \\ \hline
        Demolicious should have an example application in the form of a visual demo displayed on screen & LOW \\ \hline
        Demolicious should have a toolchain to make life easier for programmers & LOW     \\ \hline
    \end{tabular}
    \caption{Goals set for the Demolicious system}
    \label{tab:goals}
\end{table}

\newpage
\section{Structure of the Report}

\todo{Write this!}

\end{document}
