\documentclass[../main/report.tex]{subfiles}
\begin{document}

\chapter{Modern Graphics Processing Units}
\label{sec:modern_gpu}

\section{What is a GPU}
% What is a GPU?
% What is its purpose?

Modern GPUs are, in a way, an evolution of former Video Graphics Array (VGA) controllers \cite{gpu_appendix}.
A VGA controller of the early 1990s served as a memory controller and display generator that wrote framebuffer values to a display.
As technology advanced, it received hardware to perform specific graphics related functions.
This eventually evolved into a processor, with its own memory, that incorporated a full set of graphical functions.

A GPU's primary purpose has traditionally been to offload graphical calculations from the CPU and render images to a screen.
Graphical functions are accessed through APIs like DirectX and OpenGL.
Today, GPUs also have general computing capabilities and may serve as co-processors for the CPU in addition to handling their graphical duties.
Non-graphics applications for a GPU include image processing, video encoding, and many scientific computing problems and other large, highly regular calculations.


\section{The GPU Market}

% Who makes graphics accelerators?
% For what market do they make these?

% PC market reference:
% http://www.zdnet.com/latest-retail-figures-show-a-stable-pc-market-led-by-apple-and-cheap-notebooks-7000034024/
% NDP provide consumer tracking services. Link: https://www.npd.com/wps/portal/npd/us/about-npd/consumer-panel/
% This is their report: https://www.npd.com/wps/portal/npd/us/news/press-releases/apple-and-chrome-boost-us-retail-pc-back-to-school-sales/


In general, at least one GPU is present in every PC these days, and the market for PCs has been relatively stable for many years \cite{pc_sales}.
These GPUs can take the form of discrete chips or be integrated with the CPU.
Intel, AMD, and NVIDIA are the big actors in the PC GPU market \cite{gpu_overall_sales}.
Intel is largest overall but only makes GPUs integrated with their CPUs.
AMD and NVIDIA share the discrete GPU market \cite{gpu_discrete_sales}.
One of NVIDIA's GPU architectures will be used as an example later.
A big market for the more powerful discrete GPUs is the computer gaming industry.
This market has contributed a great deal to the rapid progression of graphics technologies.
With the advent of general purpose GPUs, scientific computing has also gained an interest in powerful GPUs.

In the past 6 years, the booming market for mobile smart devices has introduced a new arena for GPUs.
Mobile devices are currently outselling workstations, and every new mobile device sold today is shipped with a small-format GPU.
These GPUs are generally integrated in the device's system-on-a-chip (SoC), and have slightly different design considerations than traditional workstation GPUs.
As for everything else that runs on batteries, power consumption is a primary concern in this format.
Traditionally, this is a concern that GPU design has not needed to prioritize.
The mobile GPU format has opened up for other GPU design companies than the three mentioned above.
The three dominant GPU design companies in the SoC market are Qualcomm, ARM, and Imagination Technologies \cite{gpu_mobile_sales}.

% PC GPU market share reference:
% http://jonpeddie.com/news/comments/gpu-shipments-marketwatch-q2-2014-charts-and-images/
% and: http://jonpeddie.com/publications/add-in-board-report/
% SoC market share reference: http://jonpeddie.com/press-releases/details/qualcomm-single-largest-proprietary-gpu-supplier-imagination-technologies-t/
% "Jon Peddie Research (JPR) is a technically oriented computer graphics marketing and management-consulting firm"

\section{An Enormous Task}

Producing computer graphics is a highly processing intensive task.
Consumers expect their computers to display videos and games in high resolutions and at high framerates.
To color one pixel accurately in a 3D environment, the processor typically needs to calculate vectors in 3D space, interpolate texture data, adjust for light intensity, and more.
All of these tasks require several demanding floating point operations.

A CPU is optimized for serial programs, and will compute a picture frame in such a fashion.
The problem is that, even with the processing power of 4 GHz, it is not able to handle the amount of calculations well enough for modern graphical demands.
Fortunately, pixels can often be computed independently of each other.
A CPU fails to take advantage of this.

\section{Taking Advantage of Parallelism}

For a CPU, latency is of primary concern because the next instruction often depends on the result of the previous one.
Therefore, the CPU needs to complete each instruction as quickly as possible.
This makes the CPU very good at problems with a low level of parallelism, which after all characterizes most programs.
Finishing an instruction as quickly as possible is less of a concern when problems are more parallel in nature.
A GPU therefore optimizes for throughput instead of latency.
It gains throughput by making its architecture highly parallel in all respects.

The fast single-threaded cores in CPUs are replaced with a large quantity of smaller, slower cores.
A GPU gains throughput by spending resources on execution units instead of more cache, prediction logic, or dynamic reordering logic to make one execution unit very effective.
It also exploits parallelism with its deep pipelines, executing many instructions concurrently within each core.
These architectural decisions cause each individual thread to require more wait cycles to wait for memory access or pipeline hazards.
Fortunately, a GPU is not concerned about each individual thread's tardiness, and instead fills these wait cycles by interleaving many threads at once in each core.
Filling every cycle and every pipeline stage with useful work is, of course, essential in optimizing throughput.

% Sophisticated dynamic scheduling schemes works to maximize the utilization of GPU resources.

\section{At The Mercy of Memory}

A GPU can achieve a very high throughput because of its massive amount of parallelism.
But with great computing power comes great memory demand.
Keeping all the computational units in a GPU supplied with enough data is a difficult challenge.
Using NVIDIA's top-end GPU in 2006, GeForce 8800, as an example, it can process 32 pixels per clock at 600 MHz \cite{gpu_appendix}.
A pixel will usually be 4 bytes in size.
Calculating its value will typically require a read and write of color and depth, and a few texture element reads.
So, on average, there may be a demand of 7 memory accesses of 4 bytes each per pixel.
To sustain 32 pixel values per clock, this requires up to 896 bytes per clock, or 537 GB/s, in memory transfers.

Many techniques are used by the memory system to meet such demanding requirements.
First of all, the memory system consists of many different types of memories that are optimized for different access patterns.
Some data must be globally available to all threads, others are local to one or a handful of threads.
Some data is read only, some data is accessed with a high spatial locality in 3D space, and so on.
Tailoring the memory system to suit important and commonly used access patterns is key to high performance.
Some examples of memory types that NVIDIA use in their GPU architectures are texture memory, constant memory, local memory, and shared memory.

The different types of memories are usually organized into several separate banks, often with their own memory controller, so that more requests can be handled at a time.
Interleaving consecutive virtual addresses over different physical memory banks will assist in spreading requests evenly across the set of banks. 
All memories typically have multiple levels of cache associated with them as well.
Other techniques to improve memory bandwidth include data compression to reduce the amount of bits to transfer, and bundling memory accesses into sets that the memory system handle well.
Bundling adds latency to memory requests.
But, as mentioned earlier, this is not an issue as the GPU has plenty of other work to do while a particular thread waits.

\section{Shortcomings of a GPU}

% What kind of jobs is a GPU ill-suited for?

A GPU obviously does its parallel calculations effectively, but it comes at a cost.
It gains throughput over a CPU in part by sacrificing low latency in individual instructions.
As we have seen, this happens as a result of deep pipelines and delayed memory requests, among other things.
Such architectural features reduce the performance of serial programs, especially ones with a lot of branching.
Branching instructions on a deeply pipelined processor is heavily punished by control hazards.

Since most programs are not particularly parallelizable and commonly contain branching, a GPU is insufficient for running an average program by today's standards.
Computer users today are accustomed to the performance provided by the cooperation of a CPU and a GPU, a so-called heterogenous computer system.
This is so essential that no computer, or even mobile unit, with a graphical display is produced today without the combination of both.

\end{document}
