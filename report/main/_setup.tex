\usepackage[utf8]{inputenc}
\usepackage[T1]{fontenc} % Use 8-bit encoding that has 256 glyphs
\usepackage[english]{babel} % English language/hyphenation
\usepackage{amsmath,amsfonts,amsthm} % Math packages


\usepackage{multirow}
\usepackage{graphicx}
\usepackage{caption}
\usepackage{subcaption}

\usepackage{siunitx}
\usepackage{float}
\usepackage{subfiles} %Enables inclusions of files that can compile standalone

\usepackage{todonotes}
\usepackage{wrapfig}

\usepackage{courier}

\usepackage[nottoc]{tocbibind}

\usepackage[backend=bibtex]{biblatex}
\bibliography{bibtexlibs}

\usepackage{sectsty} % Allows customizing section commands
\allsectionsfont{\centering \normalfont\scshape} % Make all sections centered, the default font and small caps

\usepackage{array}
\newcolumntype{L}[1]{>{\raggedright\let\newline\\\arraybackslash\hspace{0pt}}m{#1}}
\renewcommand{\arraystretch}{1.5} % increase vertical spacing in table cells

\usepackage{parskip}
\setlength{\parindent}{0pt}

\usepackage{tabularx}

\usepackage{fancyhdr} % Custom headers and footers
\pagestyle{fancyplain} % Makes all pages in the document conform to the custom headers and footers
\fancyhead[L]{} % No page header - if you want one, create it in the same way as the footers below
\fancyfoot[L]{} % Empty left footer
\fancyfoot[C]{} % Empty center footer
\fancyfoot[R]{\thepage} % Page numbering for right footer
\renewcommand{\headrulewidth}{0pt} % Remove header underlines
\renewcommand{\footrulewidth}{0pt} % Remove footer underlines
\setlength{\headheight}{13.6pt} % Customize the height of the header

\usepackage{hyperref}
\hypersetup{
    colorlinks,
    citecolor=black,
    filecolor=black,
    linkcolor=black,
    urlcolor=black
}

%\usepackage{refcheck}

\fancypagestyle{firststyle}
{
  \fancyhf{}
  \fancyfoot[R]{\thepage}
}

\numberwithin{equation}{section} % Number equations within sections (i.e. 1.1, 1.2, 2.1, 2.2 instead of 1, 2, 3, 4)
\numberwithin{figure}{section} % Number figures within sections (i.e. 1.1, 1.2, 2.1, 2.2 instead of 1, 2, 3, 4)
\numberwithin{table}{section} % Number tables within sections (i.e. 1.1, 1.2, 2.1, 2.2 instead of 1, 2, 3, 4)

%------ Listings --------------
\usepackage{color}
\definecolor{light-gray}{gray}{0.95}
\definecolor{orange}{rgb}{1, 0.5, 0}
\definecolor{backcolor}{rgb}{0.95,0.95,0.92}
\usepackage{listings}
\usepackage{../main/assembly}

\lstnewenvironment{assembly}[1][]%
{\minipage{\linewidth}
\lstset{ %
language=assembly,  % choose the language of the code
basicstyle=\ttfamily\footnotesize,       % the size of the fonts that are used for the code
numbers=left,                   % where to put the line-numbers
numberstyle=\ttfamily\footnotesize,      % the size of the fonts that are used for the line-numbers
stepnumber=1,                   % the step between two line-numbers. If it is 1 each line will be numbered
resetmargins=true,              % reset line numbers
numbersep=10pt,                  % how far the line-numbers are from the code
backgroundcolor=\color{backcolor},  % choose the background color. You must add \usepackage{color}
showspaces=false,               % show spaces adding particular underscores
showstringspaces=false,         % underline spaces within strings
showtabs=false,                 % show tabs within strings adding particular underscores
tabsize=2,                      % sets default tabsize to 2 spaces
captionpos=b,                   % sets the caption-position to bottom
breaklines=true,                % sets automatic line breaking
breakatwhitespace=false,        % sets if automatic breaks should only happen at whitespace
escapeinside={\%*}{*)},         % if you want to add a comment within your code
identifierstyle=\color{blue},
stringstyle=\color{orange},
framerule=0pt,
frame=tb,
framesep=5pt,
framexleftmargin=5pt,
#1
}}%
{\endminipage}

\lstnewenvironment{c-code}[1][]%
{\minipage{\linewidth}
\lstset{ %
language=C,  % choose the language of the code
basicstyle=\ttfamily\footnotesize,       % the size of the fonts that are used for the code
numbers=left,                   % where to put the line-numbers
numberstyle=\ttfamily\footnotesize,      % the size of the fonts that are used for the line-numbers
stepnumber=1,                   % the step between two line-numbers. If it is 1 each line will be numbered
resetmargins=true,              % reset line numbers
numbersep=10pt,                  % how far the line-numbers are from the code
backgroundcolor=\color{backcolor},  % choose the background color. You must add \usepackage{color}
showspaces=false,               % show spaces adding particular underscores
showstringspaces=false,         % underline spaces within strings
showtabs=false,                 % show tabs within strings adding particular underscores
tabsize=2,                      % sets default tabsize to 2 spaces
captionpos=b,                   % sets the caption-position to bottom
breaklines=true,                % sets automatic line breaking
breakatwhitespace=false,        % sets if automatic breaks should only happen at whitespace
escapeinside={\%*}{*)},         % if you want to add a comment within your code
identifierstyle=\color{blue},
stringstyle=\color{orange},
framerule=0pt,
frame=tb,
framesep=5pt,
framexleftmargin=5pt,
#1
}}%
{\endminipage}
