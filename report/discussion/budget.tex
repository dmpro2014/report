\documentclass[../main/report.tex]{subfiles}
\begin{document}

\section{Budget}

The group was given a budget of 10 000 NOK for production of the PCB and component purchases. 
Ten boards were manufactured at the price of 10 103 NOK and components were ordered at the price of 7 571 NOK. (see appendices \ref{fig:pcb-order} and \ref{fig:component-order}).
This made for a total cost of 17 674 NOK, which results in a cost overrun of 76 \%.
There were mainly two reasons for this.

Firstly, the PCB design took so long that a fast production time was required.
This, along with a very large board, drove the cost up greatly.
Secondly, components for at least 5 PCBs were bought.
If fewer components were bought, the price would be lower.
However, this all is a part of the backup oriented design.

For this reason there are two main ways to spend less money.
The total price could be driven down by having a less backup oriented design with fewer components.
This would significantly lower the total cost.
Also, if the time limit weren't that strict, and the board could use longer time in production, the price would be lower.

\end{document}
