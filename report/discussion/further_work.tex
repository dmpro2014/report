\documentclass[../main/report.tex]{subfiles}
\begin{document}

\section{Further Work}

A beautiful thing with projects like this, is that they can be done significantly better when done the second time.
This section will focus on possible further improvements of the system.

\subsection{Hardware}
The physical design of Demolicious worked as intended.
A second version will allow for a greater energy efficiency as well as a smaller size.
All headers and unnecessary backup solutions can be removed, letting a new design focus on a small PCB with short wires and small power planes.
If a more powerful computer is to be made, a more powerful FPGA can be introduced along with a matching number of SRAMs, as a more powerful FPGA will have a greater memory need.

Furthermore, power and data can be unified in a single USB connection.
The power USB and the FPGA are currently positioned far from one another, but in a new design they can be moved closer so that the 1.2V wire that powers the FPGA can be made as short as possible.
Lastly, it would have been convenient to have had flash memory for the FPGA as this would allow for retaining the program after the power has been shut off.
This would eliminate the need to flash the FPGA after each power off.

\end{document}
