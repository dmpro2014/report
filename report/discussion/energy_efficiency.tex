\documentclass[../main/report.tex]{subfiles}
\begin{document}

\section{Energy Efficiency}

\subsection{Optimizing FPGA power usage}

While it was relatively easy to save power on the CPU side of things, the FPGA posed a bigger challenge.
The CPU easily enters deeper sleep modes, but no such thing exists by default for the Spartan-6 FPGA.
One way to reduce FPGA power consumption is to clock gate components, in effect turning them off when unneeded.
It turns out that Xilinx® supports dynamic clock gating for the Spartan-6 series used in this project \cite{xilinx-clock-gating}.
This feature can reduce power consumption up to 30\% for components opting to use chip enable signals.

The presented architecture has some usage of chip select signals, but does not shut down execution cores, instruction fetch or the LSU unit when idling.
The addition of such a system-wide chip enable signal would reduce power usage during GPU idle time, a feature that the next iteration of Demolicious should include.
This was not implemented for this project however, mainly due to time constraints.

Now for a quick calculation to see how much power can be saved in an absolute best-case scenario.
Demolicous with 8 execution cores can run the tunnel kernel at 50 FPS @ 512x256 (as seen in figure \ref{fig:kernel_tunnel_fps}), spending \SI{12.5}{ms} per frame.
When targeting 30 FPS, the GPU would spend \SI{375}{ms} calculating frames, allowing the GPU to be in a low power state for almost 62.5\% of the time.
Using a theoretical maximum of 30\% dynamic power savings for the low power state, the dynamic power consumption of \SI{0.168}{W} from figure \ref{table:scalability}, can be reduced to $ 0.70 * 0.625 * 0.168 W + 0.375 * 0.168 W = 0.1365 W$.
This represents a savings in dynamic power consumption of 18.7\%.
The total FPGA power draw would be reduced to $ 0.1365 W + 0.1860 W = 0.3225 W$, a 8.6\% reduction over the original \SI{0.3530}{W}.
This shows that FPGA power consumption still is dominated by the quiescent draw.

\subsection{Physical Implementation}

To make a PCB as energy efficient as possible, one relies upon good practices.
The wires have to be as short as possible and the board as small as possible.
This will optimize for lowest possible static power consumption.
Short wires for signal wires will make dynamic power consumption as low as possible.

The backup oriented design of the PCB doesn't fit very well in with this.
The amount of headers make wires unnecessarily long, as well as making the board itself quite large.
Headers need to make a hole through the whole board, which makes it impossible to put wires on that place.
Since the PCB has six signal layers, it means that each header removes the possibility of six wires going through where the header is.
This makes the static power consumption higher than it could be.

By having the entire bus on headers, the wires between the CPU and the GPU is longer than they could have been.
This makes the dynamic power consumption higher than if the wires are shorter.
However all of this is a trade-off as there was only one chance for the PCB to work.

\end{document}
