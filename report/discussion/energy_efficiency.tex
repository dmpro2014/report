\documentclass[../main/report.tex]{subfiles}
\begin{document}

\section{Energy Efficiency}

The system is powered by a EH-70p USB charger for the Nikon Coolpix S2700 camera \cite[p. 196]{usb-charger}.
This charger outputs a 5 V voltage and delivers 550 mA.
This gives the power input an upper bound of power consumption at $550mA * 5V = 2.75W$.

\subsection{Physical Implementation}

To make a PCB as energy efficient as possible, one relies upon good practices.
The wires have to be as short as possible and the board as small as possible.
This will optimize for lowest possible static power consumption.
Short wires for signal wires will make dynamic power consumption as low as possible.

The backup oriented design of the PCB doesn't fit very well in with this.
The amount of headers make wires unnecessarily long, as well as making the board itself quite large.
Headers need to make a hole through the whole board, which makes it impossible to put wires on that place.
Since the PCB has six signal layers, it means that each header removes the possibility of six wires going through where the header is.
This makes the static power consumption higher than it could be.

By having the entire bus on headers, the wires between the CPU and the GPU is longer than they could have been.
This makes the dynamic power consumption higher than if the wires are shorter.
However all of this is a trade-off as there was only one chance for the PCB to work.

\todo{Add actual power consumption, either based on datasheets or physical measurements?}
\todo{software related power considerations? Sleeping MCU while kernel sleeps, awsm silicon labs stuff? In addition, perhaps some elaborations on energy use inside the GPU? Identifying energy critical systems.}

\end{document}
