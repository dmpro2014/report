\chapter{GPU}

\section{Responsibilities}

\section{Kernel Structure}

\section{Receiving a Kernel Call}

\section{Running a Kernel}

\section{Architecture Overview}

\section{Component Details}

\subsection{Streaming Processor}

The streaming processor is at the heart of the GhettoCUDA architecture.
A fully operational GhettoCUDA processor will have up to 32 streaming processors wired up, allowing for an extreme degree of parallelism in code execution.

The architecture of the streaming processor is inspired by the cores of the MIPS architecture.
They consist of a single ALU, a register bank, as well as logic for reading operands from thread-private registers, shared external memory and constant storage.
Each register bank is actually composed of several register files, containing a number of general- and special purpose registers.
The register bank mirrors the interface of the register file, but only exposes a single register file at a time, which one decided by the current active barrel line.
This system allows rapid context switching between the plethora of threads co-existing inside each streaming processor.

The streaming processor support conditional execution of instructions, using a dedicated mask register to decide whether instructions should be executed or not.
This allows for branch-like behaviour without having to support the complex logic required for diverging threads.

When a memory request is invoked by the currently active instruction, memory addresses and data values are sampled from the currently active register file in the register directory, and passed on to the load/store unit.
Dedicated address and data store/return registers allows for simplification of both the programming model as well as requiring fewer wires between the set of streaming processors and the load/store unit.

\subsection{Thread Spawner}

\subsection{Register Directory}

\subsection{Load/Store Unit}

