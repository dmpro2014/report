\subsection{Load/Store Unit}
A load/store unit is responsible for handling memory requests from the core processors.
The load/store unit of Demolicious needs to be capable of simultaneously servicing incoming request from all the streaming processors.
Requests are asynchronously carried out in the background, and replies are delivered directly into the appropriate registers of the threads that made the request.

Demolicious has two independent memory banks, each capable of reading or writing one word every cycle.
To maximize the throughput to the memory, a word-striping scheme was used:
The lowest bit of the memory address determines which bank holds that location.

Because there are two independent memory banks, the incoming request are routed to two separate queues, one for each memory.
Each queue then feeds requests to its associated memory control unit.
Read responses are then handed to the write-back unit, which delivers the data to the appropriate register file.

The queues are necessary because Demolicious has more streaming processors than memory banks, so all the requests from a warp can not be completed in a single cycle.
The streaming processors work in tandem, and issue requests simultaneously.
Each memory bank can service at most one request per cycle.
As such, it takes several cycles to complete the requests for a single warp.
Since there is a limit to how many requests can be in the queue at any time, there is also a limit to how often requests can be issued.

\todo{The following does should be moved to another chapter.}
This several-cycle latency and queue size limit would have to be taken into consideration when programming for this architecture.
Loads would be not ready before several cycles later, and it would be ill-behaved to issue a request before there is room in the queue again.
To hide this latency a barrel is introduced to processor design.
A barrel makes the processor take turns executing warps from an active set in a round-robin fashion.
If the number of warps in a barrel is equal to the number of cycles of latency, the result of the first load of in the barrel can be used in the next instruction.
